\documentclass[12pt]{ctexart}
\usepackage{geometry}
\geometry{a4paper,margin=2.5cm}
\usepackage{amsmath,amssymb}
\usepackage{graphicx}
\usepackage{booktabs}
\usepackage{hyperref}
\usepackage{listings}
\usepackage{xcolor}
\usepackage{float}

\hypersetup{
  colorlinks=true,
  linkcolor=blue,
  urlcolor=blue,
  citecolor=blue
}

\lstset{
  showstringspaces=false,
  showspaces=false,
  showtabs=false,
  keepspaces=true,
  basicstyle=\ttfamily\small,
  breaklines=true,
  frame=single,
  columns=fullflexible,
  tabsize=2
}

\title{命令行、Python 与图像处理实验报告}
\author{王杰(23160001074)}
\date{\today}

\begin{document}
\maketitle

\section{实验目的}
\begin{itemize}
  \item 掌握常见 Shell 工作流:管道/重定向、作业控制、别名与配置等。
  \item 掌握 Python 基础语法:变量、流程控制、容器、函数、文件 I/O、异常、参数解析与虚拟环境。
  \item 初步使用 Python 图像处理库(Pillow、OpenCV、scikit-image、imageio)完成简单图像任务。
\end{itemize}

\section{实验环境}
\begin{itemize}
  \item 操作系统:Windows 11(WSL2: Ubuntu 24.04)
  \item Shell:\texttt{bash}(支持 \texttt{jobs}/\texttt{fg}/\texttt{bg})
  \item Python:\texttt{Python 3.11+},\texttt{pip},\texttt{venv}
  \item 第三方库:\texttt{pillow}、\texttt{opencv-python}、\texttt{scikit-image}、\texttt{imageio}、\texttt{matplotlib}
  \item \LaTeX\ 编译器:\textbf{XeLaTeX};在线:Overleaf
\end{itemize}

\section{练习内容}

\subsection{命令行环境}
\begin{enumerate}
  \item \textbf{查看命令帮助与手册}(快速定位用法)
\begin{lstlisting}[language=bash]
ls --help | head -n 10
man grep    # q 退出;/pattern 搜索;n 跳下一个
\end{lstlisting}
\begin{figure}[H]
    \centering
    \includegraphics[width=0.75\linewidth]{image1.png}
\end{figure}

  \item \textbf{重定向与管道:去重计数 Top N}
\begin{lstlisting}[language=bash]
cat /var/log/syslog 2>/dev/null \
| awk '{print $5}' | sort | uniq -c | sort -nr | head
\end{lstlisting}
\begin{figure}[H]
    \centering
    \includegraphics[width=0.75\linewidth]{image2.png}
\end{figure}

  \item \textbf{通配符与花括号扩展:批量造数据}
\begin{lstlisting}[language=bash]
mkdir demo && cd demo
touch file-{a..c}{1..3}.txt
ls file-a*.txt
\end{lstlisting}
\begin{figure}[H]
    \centering
    \includegraphics[width=0.75\linewidth]{image3.png}
\end{figure}

  \item \textbf{命令替换与子 shell}
\begin{lstlisting}[language=bash]
echo "Today is $(date +%F). Files here: $(ls | wc -l)."
( cd /etc && echo "In /etc: $(pwd)" )
\end{lstlisting}
\begin{figure}[H]
    \centering
    \includegraphics[width=0.75\linewidth]{image4.png}
\end{figure}

  \item \textbf{环境变量与别名配置}
\begin{lstlisting}[language=bash]
echo $PATH
export EDITOR=vim
alias ll='ls -alh'     # 将别名写入 ~/.bashrc 可持久化
\end{lstlisting}
\begin{figure}[H]
    \centering
    \includegraphics[width=0.75\linewidth]{image5.png}
\end{figure}

  \item \textbf{作业控制与后台运行}(Ctrl+Z 暂停,\texttt{bg}/\texttt{fg} 切换)
\begin{lstlisting}[language=bash]
sleep 1000 &
jobs                 # 查看作业
fg %1                # 回到前台
sleep 2000 & disown  # 让进程脱离终端
\end{lstlisting}
\begin{figure}[H]
    \centering
    \includegraphics[width=0.75\linewidth]{image6.png}
\end{figure}

  \item \textbf{查找并结束进程(更优雅的方式)}
\begin{lstlisting}[language=bash]
pgrep -af sleep      # 查 PID
pkill -f sleep       # 按名称结束,不手敲 PID
\end{lstlisting}
\begin{figure}[H]
    \centering
    \includegraphics[width=0.75\linewidth]{image7.png}
\end{figure}

\end{enumerate}


\subsection{Python 基础}
\begin{enumerate}
  \item \textbf{Hello, World 与 shebang}
\begin{lstlisting}[language=python]
#!/usr/bin/env python3
print("Hello, World")
\end{lstlisting}


  \item \textbf{变量与类型、f-string}
\begin{lstlisting}[language=python]
name = "Alice"; score = 92.5
print(f"{name}'s score is {score:.1f}")
\end{lstlisting}

  \item \textbf{条件与循环:奇数求和}
\begin{lstlisting}[language=python]
total = 0
for i in range(1, 100):
    if i % 2 == 1:
        total += i
print(total)
\end{lstlisting}

  \item \textbf{列表/字典推导式}
\begin{lstlisting}[language=python]
squares = [x*x for x in range(6)]
freq = {ch: "hello world".count(ch) for ch in set("hello world")}
print(squares); print(freq)
\end{lstlisting}

  \item \textbf{函数与类型注解、文档字符串}
\begin{lstlisting}[language=python]
def area(w: float, h: float) -> float:
    """Return rectangle area = w*h."""
    return w * h

print(area(3.0, 4.5))
\end{lstlisting}

  \item \textbf{文件 I/O(UTF-8)}
\begin{lstlisting}[language=python]
from pathlib import Path
Path("note.txt").write_text("第一行\n第二行\n", encoding="utf-8")
print(Path("note.txt").read_text(encoding="utf-8"))
\end{lstlisting}

  \item \textbf{异常处理(try/except/else/finally)}
\begin{lstlisting}[language=python]
try:
    x = int(input("Enter an integer: "))
except ValueError as e:
    print("Invalid!", e)
else:
    print("x^2 =", x*x)
finally:
    print("Done.")
\end{lstlisting}

  \item \textbf{命令行参数解析(\texttt{argparse})}
\begin{lstlisting}[language=python]
# save as add.py: python add.py 3 5 -> 8
import argparse
p = argparse.ArgumentParser()
p.add_argument("a", type=int); p.add_argument("b", type=int)
args = p.parse_args()
print(args.a + args.b)
\end{lstlisting}

  \item \textbf{虚拟环境与依赖导出}
\begin{lstlisting}[language=bash]
python -m venv .venv
source .venv/bin/activate   # Windows: .venv\Scripts\activate
pip install requests
pip freeze > requirements.txt
\end{lstlisting}
\end{enumerate}

\subsection{图像处理}
\begin{enumerate}
  \item \textbf{Pillow:打开/缩放/灰度化/保存}
\begin{lstlisting}[language=python]
from PIL import Image
im = Image.open("input.jpg")
im2 = im.resize((512, 512)).convert("L")
im2.save("output_pillow_gray.jpg")
\end{lstlisting}

  \item \textbf{OpenCV:高斯模糊 + Canny 边缘}
\begin{lstlisting}[language=python]
import cv2
img = cv2.imread("input.jpg")
blur = cv2.GaussianBlur(img, (5,5), 0)
edge = cv2.Canny(blur, 100, 200)
cv2.imwrite("output_cv_edge.png", edge)
\end{lstlisting}

  \item \textbf{scikit-image:Otsu 二值化}
\begin{lstlisting}[language=python]
from skimage import io, filters, img_as_ubyte
img = io.imread("input.jpg", as_gray=True)
th = filters.threshold_otsu(img)
binary = img > th
io.imsave("output_skimage_binary.png", img_as_ubyte(binary))
\end{lstlisting}

  \item \textbf{imageio:合成 GIF(配合 matplotlib 预览)}
\begin{lstlisting}[language=python]
import imageio.v2 as imageio
import glob
frames = [imageio.imread(p) for p in sorted(glob.glob("frames/*.png"))]
imageio.mimsave("anim.gif", frames, duration=0.08)
print("Saved anim.gif")
\end{lstlisting}
\end{enumerate}

\section{常见问题与解决}
\begin{itemize}
  \item \textbf{命令行作业控制失效}:确认当前 Shell 为 \texttt{bash}/\texttt{zsh},并使用 \texttt{jobs}/\texttt{fg}/\texttt{bg};脱离终端用 \texttt{disown} 或 \texttt{nohup}。
  \item \textbf{Unicode 编码问题}:统一使用 UTF-8;文件读写指定 \texttt{encoding="utf-8"}。

\section{心得体会}
通过 Missing Semester 的作业控制、别名与配置,命令行工作流显著顺滑;Python 基础结合 \texttt{argparse}/\texttt{pathlib} 等现代用法更易维护;图像处理库各擅胜场——Pillow 上手最快、OpenCV 算法丰富、scikit-image API 语义清晰、imageio 用于读写动画便捷。将三者结合,可以快速完成“命令行调度—Python 处理—结果产出”的闭环。

\section{参考资料}
\begin{itemize}
  \item 命令行环境(中文):\url{https://missing-semester-cn.github.io/2020/command-line/}
  \item Command-line Environment(英文原文):\url{https://missing.csail.mit.edu/2020/command-line/}
  \item Python 基础教程/3.x 教程(菜鸟教程):\url{https://www.runoob.com/python/python-tutorial.html},\url{https://www.runoob.com/python3/python3-tutorial.html}
  \item CSDN:【Python】推荐五个常用的图像处理库:\url{https://blog.csdn.net/sgzqc/article/details/124871774}
\end{itemize}

\section{github地址}
\url{https://github.com/tingol666/git2025q2.git}

\end{document}
