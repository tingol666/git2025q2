\documentclass[12pt]{ctexart}
\usepackage{geometry}
\geometry{a4paper,margin=2.5cm}
\usepackage{amsmath,amssymb}
\usepackage{graphicx}
\usepackage{booktabs}
\usepackage{hyperref}
\usepackage{listings}
\usepackage{xcolor}
\usepackage{float}

\hypersetup{
  colorlinks=true,
  linkcolor=blue,
  urlcolor=blue,
  citecolor=blue
}

\lstset{
  showstringspaces=false,
  showspaces=false,
  showtabs=false,
  keepspaces=true
  basicstyle=\ttfamily\small,
  breaklines=true,
  frame=single,
  columns=fullflexible,
  tabsize=2
}

\title{Shell 与 Vim 实验报告}
\author{王杰(23160001074)}
\date{\today}

\begin{document}
\maketitle

\section{实验目的}
\begin{itemize}
  \item 掌握常用 Shell 基本命令、重定向与管道。
  \item 掌握 Vim 常见编辑操作、查找替换与窗口管理。
  \item 提升命令行环境下的文件管理与文本编辑效率。
\end{itemize}

\section{实验环境}
\begin{itemize}
  \item 操作系统:Windows 11(WSL2: Ubuntu 24.04)
  \item Shell:\texttt{bash}
  \item 编辑器:\texttt{Vim 9.x}
  \item \LaTeX\ 编译器:\textbf{XeLaTeX}
  \item 在线编辑:Overleaf
\end{itemize}

\section{练习内容}

\subsection{Shell(10 个)}
\begin{enumerate}
  \item \textbf{查看当前路径与用户名}
\begin{lstlisting}[language=bash]
pwd
whoami
\end{lstlisting}
\begin{figure}[H]
    \centering
    \includegraphics[width=0.75\linewidth]{image.png}
\end{figure}
  \item \textbf{列出目录内容(普通/详细/含隐藏)}
\begin{lstlisting}[language=bash]
ls
ls -lah
\end{lstlisting}
\begin{figure}[H]
    \centering
    \includegraphics[width=0.75\linewidth]{image2.png}
\end{figure}
  \item \textbf{创建、进入与删除目录}
\begin{lstlisting}[language=bash]
mkdir lab01 && cd lab01
cd ..
rmdir lab01
\end{lstlisting}
\begin{figure}[H]
    \centering
    \includegraphics[width=0.75\linewidth]{image3.png}
\end{figure}
  \item \textbf{新建文件与查看内容}
\begin{lstlisting}[language=bash]
echo "Hello shell" > hello.txt
cat hello.txt
\end{lstlisting}
\begin{figure}[H]
    \centering
    \includegraphics[width=0.75\linewidth]{image4.png}
\end{figure}
  \item \textbf{追加内容与行数统计}
\begin{lstlisting}[language=bash]
echo "line2" >> hello.txt
wc -l hello.txt
\end{lstlisting}
\begin{figure}[H]
    \centering
    \includegraphics[width=0.75\linewidth]{image5.png}
\end{figure}
  \item \textbf{管道与过滤}
\begin{lstlisting}[language=bash]
ls /bin | head -n 5
ls -l | grep '^d'   # 仅显示目录
\end{lstlisting}
\begin{figure}[H]
    \centering
    \includegraphics[width=0.75\linewidth]{image6.png}
\end{figure}
  \item \textbf{查找文件}
\begin{lstlisting}[language=bash]
find . -type f -name "*.txt"
\end{lstlisting}
\begin{figure}[H]
    \centering
    \includegraphics[width=0.75\linewidth]{image8.png}
\end{figure}
  \item \textbf{搜索文本}
\begin{lstlisting}[language=bash]
grep -n "line" hello.txt
\end{lstlisting}
\begin{figure}[H]
    \centering
    \includegraphics[width=0.75\linewidth]{image7.png}
\end{figure}
  \item \textbf{权限与执行脚本}
\begin{lstlisting}[language=bash]
printf '#!/usr/bin/env bash\necho Hi\n' > run.sh
chmod +x run.sh
./run.sh
\end{lstlisting}
\begin{figure}[H]
    \centering
    \includegraphics[width=0.75\linewidth]{image9.png}
\end{figure}
  \item \textbf{打包与查看压缩包内容}
\begin{lstlisting}[language=bash]
tar -czf hosts.tgz /etc/hosts
tar -tzf hosts.tgz | head
\end{lstlisting}
\begin{figure}[H]
    \centering
    \includegraphics[width=0.75\linewidth]{image10.png}
\end{figure}
\end{enumerate}

\subsection{Vim(10 个)}
\begin{enumerate}
  \item \textbf{启动、保存与退出}
\begin{lstlisting}
vim notes.txt
# 在普通模式下输入:
#   :w     保存
#   :q     退出
#   :wq    保存并退出
#   :q!    强制不保存退出
\end{lstlisting}

  \item \textbf{进入插入模式与返回普通模式}
\begin{lstlisting}
# 普通模式下:
i        # 在光标前插入
a        # 在光标后插入
o / O    # 另起下一/上一行插入
<Esc>    # 返回普通模式
\end{lstlisting}

  \item \textbf{光标移动(高效导航)}
\begin{lstlisting}
h j k l  # 左 下 上 右
w / b    # 下一个/上一个单词开头
0 / $    # 行首 / 行尾
gg / G   # 文件开头 / 结尾
\end{lstlisting}

  \item \textbf{查找与高亮}
\begin{lstlisting}
/pattern    # 向下查找
n / N       # 下一个 / 上一个匹配
:set hlsearch incsearch   # 高亮与增量查找
\end{lstlisting}

  \item \textbf{替换(当前行与全文)}
\begin{lstlisting}
:s/old/new/g         # 当前行全部替换
:%s/old/new/gc       # 全文替换并逐个确认
\end{lstlisting}

  \item \textbf{撤销、重做与重复}
\begin{lstlisting}
u        # 撤销
<C-r>    # 重做 (Ctrl+r)
.        # 重复上一次操作
\end{lstlisting}

  \item \textbf{复制/粘贴/删除(操作行与词)}
\begin{lstlisting}
yy       # 复制当前行
p        # 在光标后粘贴
dd       # 删除(剪切)当前行
dw / yw  # 删除/复制一个词
x        # 删除一个字符
\end{lstlisting}

  \item \textbf{可视模式与缩进}
\begin{lstlisting}
v / V / <C-v>   # 字符/行/块可视
> / <           # 右/左缩进已选文本
=               # 自动对齐已选文本
\end{lstlisting}

  \item \textbf{分屏与窗口管理}
\begin{lstlisting}
:split file2.txt     # 水平分屏
:vsplit file3.txt    # 垂直分屏
<C-w> h/j/k/l        # 窗口间移动
:close               # 关闭当前窗口
\end{lstlisting}

  \item \textbf{多缓冲区(多文件)切换}
\begin{lstlisting}
:args a.txt b.txt c.txt   # 打开多个文件
:ls                        # 查看缓冲区
:bn / :bp                  # 下一个/上一个缓冲
:bd                        # 关闭当前缓冲
\end{lstlisting}
\begin{figure}[h]
    \centering
    \includegraphics[width=0.6\linewidth]{image11.png}
\end{figure}
\end{enumerate}

\section{常见问题与解决}
\begin{itemize}
  \item \textbf{命令找不到(command not found)}:确认是否安装并在 \texttt{PATH} 中;WSL 下可用 \texttt{sudo apt install <pkg>} 安装。
  \item \textbf{权限不足(Permission denied)}:脚本需加执行位 \texttt{chmod +x file.sh},或用 \texttt{sudo} 执行需要管理员权限的命令。
  \item \textbf{重定向覆盖文件}:使用 \texttt{>} 会覆盖,谨慎;若需追加用 \texttt{>>}。
  \item \textbf{Vim 不能保存(E212: Can't open file for writing)}:无写权限目录,使用 \texttt{:w !sudo tee \%} 或在 Shell 中 \texttt{sudo vim file}。
  \item \textbf{Vim 查找不高亮或不增量}:执行 \texttt{:set hlsearch incsearch},写入 \texttt{\~/.vimrc} 持久化。
  \item \textbf{分屏后方向键移动到中文输入法候选框}:保持普通模式下用 \texttt{<C-w> h/j/k/l} 移动窗口。
\end{itemize}

\section{心得体会}
Shell 的管道、重定向与文本处理工具能高效完成日常文件操作;Vim 的模式化编辑在熟悉常用按键后能显著提升速度。将二者结合,在终端内即可完成“查找---编辑---验证”的快速闭环;后续计划把常用设置写入 \texttt{\~/.bashrc} 与 \texttt{\~/.vimrc},进一步优化工作流。

\section{github地址}
\url{https://github.com/yourname/shell-vim-lab.git}

\end{document}
