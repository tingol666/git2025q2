\documentclass[12pt]{ctexart}
\usepackage{geometry}
\geometry{a4paper,margin=2.5cm}
\usepackage{amsmath,amssymb}
\usepackage{graphicx}
\usepackage{booktabs}
\usepackage{hyperref}
\usepackage{listings}
\usepackage{xcolor}

\hypersetup{
  colorlinks=true,
  linkcolor=blue,
  urlcolor=blue,
  citecolor=blue
}

\lstset{
  basicstyle=\ttfamily\small,
  breaklines=true,
  frame=single,
  columns=fullflexible
}

\title{Git 与 \LaTeX 实验报告}
\author{王杰(23160001074)}
\date{\today}

\begin{document}
\maketitle


\section{实验目的}
\begin{itemize}
  \item 掌握 Git 
  \item 掌握 \LaTeX
\end{itemize}

\section{实验环境}
\begin{itemize}
  \item 操作系统:Windows
  \item Git:\texttt{git}
  \item 编译器: \textbf{XeLaTeX}
  \item 编辑器: Overleaf 
\end{itemize}

\section{练习内容}

\subsection{Git}
\begin{enumerate}

  \item \textbf{初始化仓库}
        \begin{lstlisting}
mkdir test1
cd test1
git init
        \end{lstlisting}
\begin{figure}[h]
    \centering
    \includegraphics[width=0.9\linewidth]{image00.png}
\end{figure}
  \item \textbf{配置用户名邮箱}
        \begin{lstlisting}
git config --global user.name "Your Name"
git config --global user.email you@example.com
git config --list
        \end{lstlisting}
\begin{figure}[h]
    \centering
    \includegraphics[width=0.9\linewidth]{image0.png}
\end{figure}

  \item \textbf{查看状态 \& 新建文件}
        \begin{lstlisting}
echo "hello" > a.txt
git status
        \end{lstlisting}
\begin{figure}[h]
    \centering
    \includegraphics[width=0.9\linewidth]{image2.png}
\end{figure}

  \item \textbf{暂存文件}
        \begin{lstlisting}
git add a.txt
git status

        \end{lstlisting}
\begin{figure}[h]
    \centering
    \includegraphics[width=0.9\linewidth]{image3.png}
\end{figure}

  \item \textbf{提交快照}
        \begin{lstlisting}
git commit -m "add a.txt"
git log --oneline
        \end{lstlisting}
\begin{figure}[h]
    \centering
    \includegraphics[width=0.9\linewidth]{image4.png}
\end{figure}

  \item \textbf{忽略文件}
        \begin{lstlisting}
echo "*.log" > .gitignore
echo "temp.log" > temp.log
git status
        \end{lstlisting}
\begin{figure}[h]
    \centering
    \includegraphics[width=0.9\linewidth]{image.png}
\end{figure}

  \item \textbf{对比差异}
        \begin{lstlisting}
echo "world" >> a.txt
git diff
        \end{lstlisting}
\begin{figure}
    \centering
    \includegraphics[width=0.9\linewidth]{image5.png}
\end{figure}

  \item \textbf{创建与切换分支}
        \begin{lstlisting}
git checkout -b feature-1
echo "feature" >> a.txt
git add a.txt
git commit -m "feature note"
        \end{lstlisting}
\begin{figure}
    \centering
    \includegraphics[width=0.9\linewidth]{image6.png}
\end{figure}

  \item \textbf{合并分支}
        \begin{lstlisting}
git checkout master   # 或 main
git merge feature-1
git log --oneline --graph --all
        \end{lstlisting}
\begin{figure}
    \centering
    \includegraphics[width=0.9\linewidth]{image7.png}
\end{figure}

  \item \textbf{连接远程并推送}
        \begin{lstlisting}
git remote add origin <远程URL>
git push -u origin master   # 或 main
        \end{lstlisting}
\end{enumerate}

\subsection{\LaTeX(10 个)}
\begin{enumerate}
  \item \textbf{最小文档}
\begin{lstlisting}
\documentclass{article}
\begin{document}
Hello, \LaTeX!
\end{document}
\end{lstlisting}

  \item \textbf{中文文档(ctex)}
\begin{lstlisting}
\documentclass{ctexart}
\begin{document}
这是中文测试。
\end{document}
\end{lstlisting}

  \item \textbf{章节结构}
\begin{lstlisting}
\section{实验目的}
\subsection{环境}
\end{lstlisting}

  \item \textbf{数学公式}
\begin{lstlisting}
Einstein: $E=mc^2$
\[
\int_0^1 x^2\,dx = \frac{1}{3}
\]
\end{lstlisting}

  \item \textbf{有序/无序列表}
\begin{lstlisting}
\begin{enumerate}\item A \item B\end{enumerate}
\begin{itemize}\item A \item B\end{itemize}
\end{lstlisting}

  \item \textbf{插入代码(listings)}
\begin{lstlisting}
\begin{lstlisting}
git status
git commit -m "msg"
\end{lstlisting}

  \item \textbf{插入图片(graphicx)}
\begin{lstlisting}
\includegraphics[width=.5\linewidth]{logo.png}
\end{lstlisting}

  \item \textbf{插入表格}
\begin{lstlisting}
\begin{tabular}{l r}
项目 & 数值\\\hline
提交数 & 5
\end{tabular}
\end{lstlisting}

  \item \textbf{参考文献}
\begin{lstlisting}
\begin{thebibliography}{9}
\bibitem{latex} Lamport, \LaTeX.
\end{thebibliography}
\end{lstlisting}

  \item \textbf{交叉引用}
\begin{lstlisting}
\section{方法}\label{sec:method}
详见第~\ref{sec:method} 节。
\end{lstlisting}
\end{enumerate}

\section{常见问题与解决}
\begin{itemize}
  \item \textbf{中文乱码或字体缺失}:改用 XeLaTeX;文档类用 \texttt{ctexart}。
  \item \textbf{图片不显示}:确认图片文件与 \texttt{.tex} 同目录,文件名与后缀大小写一致。
  \item \textbf{Git 提交失败(没有暂存)}:先 \texttt{git add} 后 \texttt{git commit}。
  \item \textbf{合并冲突}:编辑冲突文件,保留期望内容后 \texttt{git add} 再 \texttt{git commit}。
  \item \textbf{远程拒绝}:检查权限与分支名(\texttt{main} vs \texttt{master}),首次推送用 \texttt{-u}。
\end{itemize}

\section{心得体会}
Git 部分帮助我理解了工作区、暂存区与提交历史的关系,分支合并体现了协作与并行开发的价值;\LaTeX 部分让我体会到结构化写作的效率与高质量排版的可复用性。今后计划将两者结合:用 Git 管理 \LaTeX 论文与笔记,在分支上尝试不同写作方案,并通过远程仓库进行协作与备份。

\section{github地址}
https://github.com/tingol666/git2025q2.git
\end{document}
